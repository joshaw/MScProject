% Created:  Sun 15 Jun 2014 04:25 pm
% Modified: Fri 20 Jun 2014 11:16 AM
\part{Introduction}

\section{Medical Imaging}
\label{sec:section_name}

\section[Sub-Diffraction-Limit Imaging]{Sub-Diffraction-Limit\\ Imaging}
\label{sec:sub_diffraction_limit_imaging}

Imaging objects becomes more difficult as they get smaller because of the
wavelength of light. Once two objects are separated by a distance of an order
similar to that of the wavelength ($\lambda$) of the light used to view them,
it is no longer possible to resolve these two objects apart, instead all that
can be seen is a blur of the two objects together.

There have been several techniques developed for distinguishing objects apart
on smaller and smaller scales. Many of these involve using different
wavelengths of light.  For example, instead of being limited by visible light,
$\lambda \approx 5\times 10^{-7} \textrm{m}$, x-ray radiation ($\lambda \approx
10^{-10} \textrm{m}$) or even electrons ($\lambda \approx 10^{-11} \textrm{m}$)
can be used to resolve smaller scales in x-ray and electron microscopy
respectively. These, however, have the issue that, because the smaller
wavelengths imply higher energies, there is the danger of damaging the sample.
When imaging biological samples, this can be unreasonable.
% TODO - unreasonable

\subsection{Image Manipulation}
\label{sub:image_manipulation}

Other techniques employ different methods of actually capturing the image, or
cleaver manipulation of the images that are produced, to get around the
limitations of the diffraction problem.

For example the STORM method\cite{rust2006sub} uses a technique where the
objects to be imaged are molecules of a fluorescent dye. The type of dye
molecule used allows the fluorescence to be switched on and off, allowing some
markers to be imaged separately to others, effectively increasing the distance
between points. Once an image is captured, the point spread function (PSF) of
the point is used to locate the single marker, the ``on'' markers are changed
and the image retaken.

% TODO - place somewhere sensible
Even using exotic types of lenses to reduce or remove the problem of
diffraction\cite{fang2005sub}.
