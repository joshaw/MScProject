% Created:  Wed 13 Aug 2014 04:53 PM
% Author:   Josh Wainwright
% Filename: requirements.tex

\onecolumn
\section{Requirements}
\label{sec:requirements}

A number of requirements that should be achieved during this project are listed
below. Each is given a value on the ``MoSCoW'' scale, meaning `M' is a feature
that \textbf{M}ust be included in order for the final product to fulfill the
brief in Part~\ref{part:imagej_plugin}, `S' is a feature that \textbf{S}hould
be included, `C' is a feature that \textbf{C}ould be included to increase the
usability and features but is not essential and `W' is something that
\textbf{W}ould be included if there is time and resources to do so.

\subsection{Functional Requirements}
\label{sub:functional_requirements}

The plugin will allow the user to:

\begin{enumerate}[label=\arabic*.]
	\item Select data:
		\begin{enumerate}[label*=\arabic*.]
			\item~\label{req:a} Select a data file to process. (M)
			\item~\label{req:b} Select the appropriate column separator for the
				file. (C)
			\item~\label{req:c} Select which column from the data file the x-
				and y-coordinates appear in. (W)
			\item~\label{req:d} Adjust parameters relating to the process of
				analysing the data file. (S)
		\end{enumerate}
	\item Create images:
		\begin{enumerate}[label*=\arabic*.]
			\item~\label{req:e} Create an image of the points from the file
				using native ImageJ functionality. (S)
			\item~\label{req:f} Create an image of the clusters found using
				native ImageJ functionality. (M)
		\end{enumerate}
	\item Clustering algorithm:
		\begin{enumerate}[label*=\arabic*.]
			\item~\label{req:g} Perform a clustering algorithm on the data in
				the chosen file. (M)
			\item~\label{req:ga} Choose from alternative clustering algorithms
				to perform on the data. (C)
		\end{enumerate}
	\item Generate cluster information:
		\begin{enumerate}[label*=\arabic*.]
			\item~\label{req:h} Generate perimeter information for each of the
				clusters found. (S)
			\item~\label{req:i} Generate area information for each of the
				clusters found. (S)
			\item~\label{req:j} Display a results table showing a summary of
				information about each of the clusters found. (C)
			\item~\label{req:k} Limit which of the clusters are drawn to the
				image based on their size. (C)
			\item~\label{req:l} Limit which of the clusters are included in the
				results table based on the size of the cluster. (C)
		\end{enumerate}
	\item Export data found by the algorithm:
		\begin{enumerate}[label*=\arabic*.]
			\item~\label{req:m} Export information by selecting appropriate
				cluster from the results table. (W)
			\item~\label{req:n} Export data points contained in cluster by
				selecting appropriate cluster from the results table. (W)
		\end{enumerate}
\end{enumerate}

\subsection{Non-Functional Requirements}
\label{sub:non_functional_requirements}

The plugin will:

\begin{enumerate}[label=\arabic*.]
	\item Efficiency:
		\begin{enumerate}[label*=\arabic*.]
			\item~\label{req:o} Handle input data files of up to \num{100000}
				data points in under 10 seconds. (S)
		\end{enumerate}
	\item Usability:
		\begin{enumerate}[label*=\arabic*.]
			\item~\label{req:p} Be easy to use without instruction. (M)
			\item~\label{req:q} Have customisation ability for more advanced
				uses. (S)
		\end{enumerate}
	\item Interoperability:
		\begin{enumerate}[label*=\arabic*.]
			\item~\label{req:r} Be platform independent, as long as ImageJ is
				present. (S)
		\end{enumerate}
\end{enumerate}

\restoregeometry%
