% Created:  Wed 13 Aug 2014 04:53 PM
% Author:   Josh Wainwright
% Filename: requirements.tex

\onecolumn
\section{Requirements}
\label{sec:requirements}

% TODO Requirements intro

\subsection{Functional Requirements}
\label{sub:functional_requirements}

The plugin will allow the user to:

\begin{enumerate}
	\item Select data
		\begin{enumerate}
			\item Select a data file to process.
			\item Select the appropriate column separator for the file.
			\item Select which coloumn the x- and y-coordinates appear in.
			\item Adjust parameters relating to the process of analysing the
				data file.
		\end{enumerate}
	\item Create images
		\begin{enumerate}
			\item Create an image of the points from the selected file using
				naitve ImageJ functionality.
			\item Create an image of the clusters found using native ImageJ
				functionality.
		\end{enumerate}
	\item Perform a clustering algorithm on the data in the chosen file.
	\item Generate cluster information
		\begin{enumerate}
			\item Generate perimeter information for each of the clusters
				found.
			\item Generate area information for each of the clusters found.
			\item Display a results table showing summary of information about
				each of the clusters found.
			\item Limit the clusters drawn to the image based on the size of
				the cluster.
			\item Limit the clusters included in the results table based on the
				size of the cluster.
		\end{enumerate}
	\item Export data found by the algorithm.
		\begin{enumerate}
			\item Export cluster information by selecing appropriate cluster
				from the results table.
			\item Export data points contained in cluster by selecing
				appropriate cluster from the results table.
		\end{enumerate}
\end{enumerate}

\subsection{Non-Functional Requirements}
\label{sub:non_functional_requirements}

The plugin will:

\begin{enumerate}
	\item Efficiency
		\begin{enumerate}
			\item Handle input data files of upto \num{100000} data points in
				under 10 seconds.
		\end{enumerate}
	\item Usability
		\begin{enumerate}
			\item Be easy to use without instruction.
			\item Have customisation ability for more advanced uses.
		\end{enumerate}
	\item Interoperability
		\begin{enumerate}
			\item Be platform independant, as long as ImageJ is present.
		\end{enumerate}
\end{enumerate}

\restoregeometry
