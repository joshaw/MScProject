% Created:  Wed 09 Jul 2014 03:00 PM
% @author Josh Wainwright
% File name : clusters.tex

\part{Cluster Analysis}
\label{prt:cluster_analysis}

Cluster analysis is the grouping of a set of objects or items in a spatially or
informationally logical way such that the items that are placed in the same
group are more similar to each other than they are to the objects in the other
groups in the set. These groups shall be called \emph{clusters}. When dealing
with images, the clustering that is of interest is based on spatial location,
i.e., clusters should be composed of objects that are close together in the
image and clusters should be separated by regions of emptiness or background
level noise.

One of the primary reasons for chosing the quadtree method over the simple grid
methods is that the simple act of placing the objects, in this case coordinates
of data points, into the quadtree starts the process of analysing the data.
Since the points end up in a tree structure with the number of points closely
separated being on the lowest levels of the tree, the data is already clustered
in a way.

There are a number of alternative methods of identifying clusters in images.
