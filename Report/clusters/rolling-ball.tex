% Created:  TIMESTAMP
% Modified: TIMESTAMP

\subsection{Rolling Ball Analysis}
\label{sub:rolling_ball_analysis}

The accessible surface area algorithm, also known as the "Rolling Ball Method"
is a technique in image processing for describing the outer limit of a cluster
of points. It is derived from biological molecules analysis where the surface
area of a molecule that is accessible to a solvent.

The rolling ball method can be used to analyse a cluster of points by imagining
a ball that sits against one of the outer-most points. From here it is "rolled"
around the cluster such that it is always touching at least one point.

Once the ball has reached the point it started at, the line that the ball
traced is reduced in size by the radius of the ball. This line then represents
the outer limit of the cluster.

The size of the ball must be chosen depending on the average separation of the
points within the cluster.
