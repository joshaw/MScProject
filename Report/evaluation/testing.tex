% Created:  Mon 01 Sep 2014 01:03 PM
% Author:   Josh Wainwright
% Filename: testing.tex

\section{Testing}
\label{sec:testing}

Since the absolute definition of the clusters that the plugin looks for are
somewhat subjective, testing the algorithm for correctness is difficult. The
individual functions and smaller components of the algorithms are tested using
the unit testing framework JUnit\cite{tahchiev2010junit}, and the overall
effectiveness of the clustering algorithms is validated with generated data
files and acceptance testing.

\subsection{Generated Test Files}
\label{sub:generated_test_files}

% TODO

\subsection{Volunteer Validation}
\label{sub:volunteer_validation}

The first stage of acceptance testing took the form of a sample set of data
that was plotted onto an image using the discrete grid method. This data was
then presented to a number of volunteers who were asked to count the number of
clusters they could identify and then highlight the clusters that they were
able to observe in order of precedence, most defined through to least defined.

Once the volunteer had identified the clusters, the algorithm was performed on
the same set of data and the results compared to the predicted results.
Finally, when the algorithm identified clusters that the volunteers had not,
they were asked to identify which of these they did not consider to be valid
clusters and which were valid clusters that they had missed. The number of
false positive clusters identified by the quadtree algorithm was found, by this
method, to be low.

When using the settings described in Section~\ref{sub:option_sliders}, 100\% of
the clusters identified by the volunteers were able to be located by the
algorithm. Of the clusters that the algorithm found but the volunteers did not,
around 60\% were considered to be valid by the volunteers. These results
suggest that the algorithm is performing well at being able to identify
clusters, but requires some intervention to get the best results (see
Section~\ref{sec:possible_improvements}) and that, in general, the algorithm
does not produce that many false positive results.

\subsection{Researcher Validation}
\label{sub:researcher_validation}

TODO

The final stage of acceptance testing was performed to confirm the type of
clusters identified conformed with the actual objects that are being imaged.
This was performed by two researchers working in the medical imaging field who
produced the sample data used in this project.
% TODO
