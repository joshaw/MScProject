% Created:  Wed 13 Aug 2014 04:57 PM
% Author:   Josh Wainwright
% Filename: requirements.tex

\section{Requirements Satisfaction}
\label{sec:requirements_satisfaction}

With the plugin written, the requirements outlined in
Section~\ref{sec:requirements} are examined to check which have been fulfilled
and which were not able to be completed.

\subsection{Functional Requirements}
\label{sub:functional_requirements}

\begin{description}[style=unboxed]
	\item[\ref{req:a} Select a data file to process.] \emph{Completed} with
		file chooser dialogue.

	\item[\ref{req:b} Select the appropriate column separator for the file.]
		\emph{Completed} with column chooser integrated in the file chooser.

	\item[\ref{req:c} Select which coloumn the x- and y-coordinates appear in.]
		\emph{Completed} with column chooser integrated in the file chooser.

	\item[\ref{req:d} Adjust parameters relating to the process of analysing
		the data file.] \emph{Completed} with sliders on main user interface.

	\item[\ref{req:e} Create an image of the points from the selected file
		using naitve ImageJ functionality.] \emph{Completed} with simple grid
		implementation which just adds the points to the grid and draws the
		image to the screen, no analysis of the clusters is performed.

	\item[\ref{req:f} Create an image of the clusters found using native ImageJ
		functionality.] \emph{Completed} with ImageJ image of clusters found,
		colour-coded based on order the clusters were found.

	\item[\ref{req:g} Perform a clustering algorithm on the data in the chosen
		file.] \emph{Completed} with the quadtree based clustering algorithm.

	\item[\ref{req:h} Generate perimeter information for each of the clusters
		found.] \emph{Completed} with dilate/erode based perimeter method.

	\item[\ref{req:i} Generate area information for each of the clusters
		found.] \emph{Completed} with dilate/erode based area method.

	\item[\ref{req:j} Display a results table showing summary of information
		about each of the clusters found.] \emph{Completed} with ImageJ results
		table shown after cluster analysis is complete.

	\item[\ref{req:k} Limit the clusters drawn to the image based on the size
		of the cluster.] \emph{Completed} with limit chosen by user from main
		user interface.

	\item[\ref{req:l} Limit the clusters included in the results table based on
		the size of the cluster.] \emph{Completed} with limit chosen by user
		from main user interface.

	\item[\ref{req:m} Export cluster information by selecing appropriate
		cluster from the results table.] \emph{Not completed} since integration
		with the results table (which is defined and maintained by ImageJ) does
		not allow selecting a line and passing that line to the plugin.

	\item[\ref{req:n} Export data points contained in cluster by selecing
		appropriate cluster from the results table.] \emph{Not completed} since
		ImageJ does not allow making clickable selections from images as this
		would interfere with manipulation of the image itself.
\end{description}

\subsection{Non-Functional Requirements}
\label{sub:non_functional_requirements}

\begin{description}
	\item[\ref{req:o} Handle input data files of upto 100\,000 data points in
		under 10 seconds.] \emph{Completed}, though the time to process files
		is very much dependant on the hardware used, the algorithms developed
		are quite efficient.

	\item[\ref{req:p} Be easy to use without instruction.] \emph{Completed}
		with a small number of test users who were allowed to use the plugin
		and comment on its usability.

	\item[\ref{req:q} Have customisation ability for more advanced uses.]
		\emph{Not completed}, a number of options are provided to customise the
		running of the algorithm but these are required to work and so there is
		no other customisation.

	\item[\ref{req:r} Be platform independant, as long as ImageJ is
		present.] \emph{Completed} with ImageJ and Java being platform
		independent so the plugin should work wherever ImageJ and Java is
		installed.
\end{description}
