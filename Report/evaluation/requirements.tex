% Created:  Wed 13 Aug 2014 04:57 PM
% Author:   Josh Wainwright
% Filename: requirements.tex

\section{Requirements Satisfaction}
\label{sec:requirements_satisfaction}

With the plugin written, the requirements outlined in
Section~\ref{sec:requirements} are examined to check which have been fulfilled
and which were not able to be completed. Each of the requirements below refer
to a requirement from Section~\ref{sec:requirements}. In each case, the
requirement is given with a note to say whether it was completed and if so,
what component is the fulfillment of it and if not, why not.

\subsection{Functional Requirements}
\label{sub:functional_requirements}

\begin{enumerate}[label=\arabic*.]
	\item Select data:
		\begin{enumerate}[label*=\arabic*.]
			\item~\label{req:a} Select a data file to process. (M) \\
 	 	 	\emph{Completed} with file chooser dialogue.

			\item~\label{req:b} Select the appropriate column separator for the
				file. (C) \\
				\emph{Completed} with column chooser integrated into the file
				chooser.

			\item~\label{req:c} Select which column from the data file the x-
				and y-coordinates appear in. (W) \\
				\emph{Completed} with column chooser integrated into the file
				chooser.

			\item~\label{req:d} Adjust parameters relating to the process of
				analysing the data file. (S) \\
				\emph{Completed} with sliders and checkboxes for options in
				main user interface.

		\end{enumerate}
	\item Create images:
		\begin{enumerate}[label*=\arabic*.]
			\item~\label{req:e} Create an image of the points from the file
				using native ImageJ functionality. (S) \\
				\emph{Completed} with simple grid implementation which simply
				adds the points to the grid and draws the image to the screen,
				no analysis of the clusters is performed.

			\item~\label{req:f} Create an image of the clusters found using
				native ImageJ functionality. (M) \\
				\emph{Completed} with an ImageJ image of clusters found,
				colour-coded based on order in which the clusters were found.

		\end{enumerate}
	\item Clustering algorithm:
		\begin{enumerate}[label*=\arabic*.]
			\item~\label{req:g} Perform a clustering algorithm on the data in
				the chosen file. (M) \\
				\emph{Completed} with the quadtree based clustering algorithm.

			\item~\label{req:ga} Choose from alternative clustering algorithms
				to perform on the data. (C) \\
				\emph{Not completed} The simple grid method was implemented but
				found to perform too badly to be used as an alternative
				clustering algorithm. Limitations in time prevented the
				implementation of existing clustering algorithms.

		\end{enumerate}
	\item Generate cluster information:
		\begin{enumerate}[label*=\arabic*.]
			\item~\label{req:h} Generate perimeter information for each of the
				clusters found. (S) \\
				\emph{Completed} with dilate/erode or concave hull based
				perimeter methods.

			\item~\label{req:i} Generate area information for each of the
				clusters found. (S) \\
				\emph{Completed} with dilate/erode or concave hull based area
				methods.

			\item~\label{req:j} Display a results table showing a summary of
				information about each of the clusters found. (C) \\
				\emph{Completed} with ImageJ results table shown after cluster
				analysis is completed.

			\item~\label{req:k} Limit which of the clusters are drawn to the
				image based on their size. (C) \\
				\emph{Completed} with limit chosen by user from main user
				interface.

			\item~\label{req:l} Limit which of the clusters are included in the
				results table based on the size of the cluster. (C) \\
				\emph{Completed} with limit chosen by user from main user
				interface.

		\end{enumerate}
	\item Export data found by the algorithm:
		\begin{enumerate}[label*=\arabic*.]
			\item~\label{req:m} Export information by selecting appropriate
				cluster from the results table. (W) \\
				\emph{Not completed} since integration with the results table
				(which is defined and maintained by ImageJ) does not allow
				selecting a line and passing that line to the plugin. To
				implement this functionality would require a custom
				implementation of the results table.

			\item~\label{req:n} Export data points contained in cluster by
				selecting appropriate cluster from the results table. (W) \\
				\emph{Not completed} since ImageJ does not allow making
				clickable selections from images as this would interfere with
				manipulation of the image itself.

		\end{enumerate}
\end{enumerate}

\subsection{Non-Functional Requirements}
\label{sub:non_functional_requirements}

\begin{enumerate}[label=\arabic*.]
	\item Efficiency:
		\begin{enumerate}[label*=\arabic*.]
			\item~\label{req:o} Handle input data files of up to \num{100000}
				data points in under 10 seconds. (S) \\
				\emph{Completed}, though the time to process files is very much
				dependant on the hardware used, the algorithms developed are
				quite efficient.

		\end{enumerate}
	\item Usability:
		\begin{enumerate}[label*=\arabic*.]
			\item~\label{req:p} Be easy to use without instruction. (M) \\
				\emph{Completed} with a small number of test users who were
				allowed to use the plugin and comment on its usability.

			\item~\label{req:q} Have customisation ability for more advanced
				uses. (S) \\
				\emph{Not completed}, a number of options are provided to
				customise the running of the algorithm but these are required
				to work and so there is no other customisation.

		\end{enumerate}
	\item Interoperability:
		\begin{enumerate}[label*=\arabic*.]
			\item~\label{req:r} Be platform independent, as long as ImageJ is
				present. (S) \\
				\emph{Completed} with ImageJ and Java being platform
				independent so the plugin should work wherever ImageJ and Java
				are installed.

		\end{enumerate}
\end{enumerate}
