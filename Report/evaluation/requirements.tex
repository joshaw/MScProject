% Created:  Wed 13 Aug 2014 04:57 PM
% Author:   Josh Wainwright
% Filename: requirements.tex

\section{Requirements Satisfaction}
\label{sec:requirements_satisfaction}

With the plugin written, the requirements outlined in
Section~\ref{sec:requirements} are examined to check which have been fulfilled
and which were not able to be completed. Each of the requirements below refer
to a requirement from Section~\ref{sec:requirements}. In each case, the
requirement is given with a note to say whether it was completed and if so,
what component is the fulfillment of it and if not, why not.

\subsection{Functional Requirements}
\label{sub:functional_requirements}

\begin{description}[style=unboxed]
	\item[\ref{req:a} Select a data file to process.]
		\hfill \\ \emph{Completed} with file chooser dialogue.

	\item[\ref{req:b} Select the appropriate column separator for the file.]
		\hfill \\ \emph{Completed} with column chooser integrated into the file
		chooser.

	\item[\ref{req:c} Select which coloumn the x- and y-coordinates appear in.]
		\hfill \\ \emph{Completed} with column chooser integrated into the file
		chooser.

	\item[\ref{req:d} Adjust parameters relating to the process of analysing
		the data file.]
		\hfill \\ \emph{Completed} with sliders and checkboxes for options in
		main user interface.

	\item[\ref{req:e} Create an image of the points from the selected file
		using naitve ImageJ functionality.]
		\hfill \\ \emph{Completed} with simple grid implementation which simply
		adds the points to the grid and draws the image to the screen, no
		analysis of the clusters is performed.

	\item[\ref{req:f} Create an image of the clusters found using native ImageJ
		functionality.]
		\hfill \\ \emph{Completed} with an ImageJ image of clusters found,
		colour-coded based on order in which the clusters were found.

	\item[\ref{req:g} Perform a clustering algorithm on the data in the chosen
		file.]
		\hfill \\ \emph{Completed} with the quadtree based clustering
		algorithm.

	\item[\ref{req:ga} Choose from alternative clustering algorithms
		to perform on the data.]
		\hfill \\ \emph{Not completed} The simple grid method was implemented
		but found to perform too badly to be used as an alternative clustering
		algorithm. Limitations in time prevented the implementation of existing
		clustering algorithms.

	\item[\ref{req:h} Generate perimeter information for each of the clusters
		found.]
		\hfill \\ \emph{Completed} with dilate/erode or concave hull based
		perimeter methods.

	\item[\ref{req:i} Generate area information for each of the clusters
		found.]
		\hfill \\ \emph{Completed} with dilate/erode or concave hull based area
		methods.

	\item[\ref{req:j} Display a results table showing summary of information
		about each of the clusters found.]
		\hfill \\ \emph{Completed} with ImageJ results table shown after
		cluster analysis is completed.

	\item[\ref{req:k} Limit the clusters drawn to the image based on the size
		of the cluster.]
		\hfill \\ \emph{Completed} with limit chosen by user from main user
		interface.

	\item[\ref{req:l} Limit the clusters included in the results table based on
		the size of the cluster.]
		\hfill \\ \emph{Completed} with limit chosen by user from main user
		interface.

	\item[\ref{req:m} Export cluster information by selecting appropriate
		cluster from the results table.]
		\hfill \\ \emph{Not completed} since integration with the results table
		(which is defined and maintained by ImageJ) does not allow selecting a
		line and passing that line to the plugin. To implement this
		functionality would require a custom implementation of the results
		table.

	\item[\ref{req:n} Export data points contained in cluster by selecing
		appropriate cluster from the results table.]
		\hfill \\ \emph{Not completed} since ImageJ does not allow making
		clickable selections from images as this would interfere with
		manipulation of the image itself.
\end{description}

\subsection{Non-Functional Requirements}
\label{sub:non_functional_requirements}

\begin{description}
	\item[\ref{req:o} Handle input data files of up to 100\,000 data points in
		under 10 seconds.]
		\hfill \\ \emph{Completed}, though the time to process files is very
		much dependant on the hardware used, the algorithms developed are quite
		efficient.

	\item[\ref{req:p} Be easy to use without instruction.]
		\hfill \\ \emph{Completed} with a small number of test users who were
		allowed to use the plugin and comment on its usability.

	\item[\ref{req:q} Have customisation ability for more advanced uses.]
		\hfill \\ \emph{Not completed}, a number of options are provided to
		customise the running of the algorithm but these are required to work
		and so there is no other customisation.

	\item[\ref{req:r} Be platform independent, as long as ImageJ is
		present.]
		\hfill \\ \emph{Completed} with ImageJ and Java being platform
		independent so the plugin should work wherever ImageJ and Java are
		installed.
\end{description}
