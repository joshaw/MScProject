% Created:  Tue 01 Jul 2014 03:19 PM
% Modified: Wed 06 Aug 2014 11:27 am
% @author Josh Wainwright
% File name : datastructures.tex

\part{Data Structures}
\label{prt:data_structures}

The way in which data is represented in memory has a large effect on the speed,
efficiency and effectiveness of any algorithm that is performed on that data.
There are almost always a number of tradeoffs that must be considered when
choosing or designing a data structure: speed of access vs.\ speed of search or
traversal, storage space used vs.\ time to insert a datum, etc.

Some data structures may be na\"ively chosen based on one of these, at the
expence of the other. For example, consider storing the pixel information for a
sparse image generated by a number of single pixel points---a linear array
might be selected. This would give extremely good access and modification
times, both $O(1)$, but insertion and deletion are very slow. There would also
be a large amount of wasted space since every pixel that is black, i.e., does
not have any points in, would need to have an array index with the same entry,
$0$.

To decide the most approriate data structure, a number of different approaches
are implemented and tested under different types of data and for a range of
different operations performed on them.
