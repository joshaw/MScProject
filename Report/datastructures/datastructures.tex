% Created:  Tue 01 Jul 2014 03:19 PM
% @author Josh Wainwright
% File name : datastructures.tex

\part{Custom Algorithm}
\label{prt:custom_algorithm}

The way in which data is represented in memory has a large effect on the speed,
efficiency and effectiveness of any algorithm that is performed on that data.
There are always a number of trade-offs that must be considered when choosing
or designing a data structure for an algorithm: speed of access vs.\ speed of
search, storage space used vs.\ time to insert a datum, etc.

Some data structures may be na\"{\i}vely chosen based on one of these, at the
expense of the other. For example, consider storing the pixel information for a
sparse image generated by a number of single pixel points---a linear array
might be selected. This would give extremely good access and modification
times, both constant, $O(1)$, but insertion and deletion are slow, $O(n)$.
There would also be a large amount of wasted space since every pixel that is
black, i.e., does not have any points in, would need to have an array index
with the same entry, $0$.

To decide the most appropriate data structure, a number of different approaches
are implemented and tested under different types of data and for a range of
different operations performed on them.
