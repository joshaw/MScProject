% Created:  Thu 24 Jul 2014 04:40 PM
% @author Josh Wainwright
% filename: improvements.tex

\section{Possible Improvements}
\label{sec:possible_improvements}

Though the primary  aims of this project were fulfilled, there are a few areas
that could be improved, or features that could be added that would improve the
plugin and make it easier to use. These features were not developed during the
course of this project because the core functionality was more important and
time constraints prevented them from being added after the existing features
were complete.

\begin{description}

	\item[Alternative algorithms] \hfill

		The algorithm that was developed for this project performs well under a
		range of different conditions. This was tested through the use of
		exemplar data sets and data sets constructed for use in testing to
		simulate different conditions. However, there may be some situations
		that were not tested for which it might not perform well. To avoid
		this, a number of alternative algorithms could be included so that, if
		the default algorithm fails, the user could switch to a different
		approach which might handle the data better.

		One or more of the existing algorithms discussed in
		Part~\ref{prt:existing_cluster_analysis_algorithms} could be
		implemented as alternatives which could then be selected from the main
		user interface if the quadtree based method failed.

	\item[Better starting location selection] \hfill

		The logic in the current implementation that selects new starting
		locations where clusters are propagated from is reasonably simple and
		so can fail to determine the best next location. This would be improved
		by a method that considered more than simply the quadtree structure,
		such as the placement of existing clusters and the depth of node
		surrounding the possible starting location.

	\item[Better algorithm halting] \hfill

		Similarly to above, the rules that stops the algorithm looking for more
		starting locations once a cluster has finished being propagated are not
		always sufficient. For example, in a data set with few clusters, from
		zero to six, the algorithm will continue to look for more and so might
		identify the background noise as a cluster that spans most of the
		image.

	\item[Calculation of optimum settings] \hfill

		Since the particular settings that are required vary a lot between
		different data sets, the user must often alter the settings before any
		useful results are found. To automate this process would require some
		significant changes to the algorithm so that it did not take a
		prohibitive amount of time to produce results, but would make the
		plugin far easier to use.

	\item[Use of additional fields from data file] \hfill

		The STORM and PALM imaging processes produce a considerable amount of
		data more than just the x- and y- coordinate information that is used
		by the plugin. A summary of the headers of the columns from a file is
		given in Appendix~\ref{app:data_file_structure}. A number of these
		extra parameters for each data point could be used to enhance the
		algorithms to produce better results. For example, the \texttt{width}
		field, which represents the full width at half maximum of the point,
		could be used to classify the data point and maybe give some indication
		about whether it is interesting or is noise.

\end{description}
