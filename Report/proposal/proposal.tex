\section{Background}
\label{sec:background}

In medical imaging, viewing individual features of single cells is essential to
learn about how the processes in the cells take place, but also extremely
difficult. To view something as small as a cell, fluorescent dyes are attached
to the cell. When light of a certain frequency is shone on these dyes, the dye
molecule is excited. When de-excitation occurs, a photon of light is released
with a different wave length. This allows the dye molecules to be imaged.

Imaging objects becomes more difficult as they get smaller because of the
wavelength of light. Once two objects are separated by a distance of an order
similar to that of the wavelength ($\lambda$) of the light used to view them,
it is no longer possible to resolve these two objects apart, instead all that
can be seen is a blur of the two objects together. The dyes that are used
usually respond in the visible frequency range, thus limiting the resolution of
separate points to around \SI{300}{\nano\metre}.

There are a number of techniques that have been developed to avoid this
problem. Some of them employ different wavelengths of light. X-ray and electron
microscopy use x-rays and electrons respectively which have shorter wavelengths
than visible light and so can resolve smaller distances. These have problems,
however, since the shorter wavelengths imply higher energies and so often
cause damage or require destroying to the sample.

Two examples that are used in medical imaging are PALM\cite{owen2010palm} and
STORM\cite{rust2006sub}. These both employ special dyes that allow molecules to
be imaged at different times, a different subset of all the dye molecules are
activated for each image. Using the point spread function (PSF) of the imaged
molecule, the precise location can be estimated and combined with the locations
of other molecules from other images of the cell. This allows the diffraction
limit to be circumvented without affecting the sample material, producing a
large number of points, each representing a single molecule attached to a
position on the cell.

\section{Project}
\label{sec:project}

This project aims to interpret the data that is produced by this analysis in a
more efficient way than currently used methods and to provide certain
quantitative statistics regarding the data; size of structure, number of
points, density of points, etc. Current analysis methods have proven to be
inefficient and are often not able to cope well with noisy data. This project
will investigate alternative methods for identifying clusters in the data and
ignoring erroneous points.

\subsection{Uniform Discrete Cell Method}
\label{sub:uniform_discrete_cell_method}

An initial method for identifying structure in the data points will involve
splitting the image space into discretized square cells and treating each cell
as a grey-scale pixel. The value that will be assigned to each pixel will
depend on the number of data points that it contains and noise will be removed
by thresholding the image to a predetermined limit value. This will require
defining a cell size before starting and then analysing the image space as a
whole.

\subsection{Quadtree Method}
\label{sub:quadtree_method}

A second method that will be investigated will be to use a quadtree abstract
data type to arrange the data points such that structure emerges naturally.
Each datum will be placed into the quadtree such that only up to a maximum
number of points in each node is allowed. This should have the benefit of being
much more dynamic, so being faster and less resource demanding to achieve. It
will require that the data structure that is used can be traversed efficiently
in order to determine adjacent nodes when extracting statistics.

With both of these initial approaches, a method for classifying the resultant
information will need to be produced and tested. These strategies will be
adapted through the development and testing stages depending on their
performance with large real data-sets.

\section{Deliverables}
\label{sec:deliverables}

The final goal deliverable will be an easily usable and intuitive plugin for
the ImageJ program. ImageJ\cite{rasband1997imagej}, developed by the Natural
Institutes of Health, is used as the industry standard for analysis and
manipulation of biological or medical images. Since it is a public-domain
program with an open Java plugin architecture, this should be achievable.

\section{Software Development Model}
\label{sec:software_development_model}

There is a distinction in steps between the initial design of the data
structres and algorithms and the development of the program and the plugin for
ImageJ. To make the most use of this, the project will follow an Agile
development model. There will be stages of development of the algorithms and
the plugin and these will be revisited as necessary during the development.

\section{Preliminary Timescale}
\label{sec:preliminary_timescale}

\renewcommand{\arraystretch}{1.3}
\begin{tabu} to \linewidth {c X l}
	\toprule
	Stage & Tasks & Date \\
	\midrule
	1  & Research existing methods of cluster analysis and identify short comings.         & 20th June 2014 \\
	2  & Build implementation of Uniform Discrete Cell method.                             & 27th June 2014 \\
	3  & Build implementation of Quadtree method.                                          & 11th July 2014 \\
	4  & Test and compare previous algorithms. Perform timing and resource usage analysis. & 18th July 2014 \\
	5  & Build first iteration ImageJ plugin using chosen method.                          & 25th July 2014 \\
	6  & Using chosen method, implement cluster analysis algorithms.                       & 1th August 2014 \\
	7  & Add cluster analysis to ImageJ plugin.                                            & 8th August 2014 \\
	8  & Performance and ease of use testing of plugin.                                    & 15th August 2014 \\
	8  & Write-up of background research and current implementations investigation         & 18th July 2014 \\
	10 & Write-up of data structure algorithms.                                            & 1st August 2014 \\
	11 & Write-up of cluster analysis algorithms.                                          & 22nd August 2014 \\
	12 & Final write-up of processes, improvements and end results of project.             & 29th August 2014 \\
	\bottomrule
\end{tabu}
