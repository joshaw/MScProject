\section{Background}
\label{sec:background}

In medical imaging, viewing individual features of single cells is essential to
learn about how the processes in the cells take place, but also extremely
difficult. To view something as small as a cell, fluorescent dyes are attached
to the cell. When light of a certain frequency is shone on these dyes, the dye
molecule is excited. When de-excitation occurs, a photon of light is released
with a different wave length. This allows the dye molecules to be imaged.

Imaging objects becomes more difficult as they get smaller because of the
wavelength of light. Once two objects are separated by a distance of an order
similar to that of the wavelength ($\lambda$) of the light used to view them,
it is no longer possible to resolve these two objects apart, instead all that
can be seen is a blur of the two objects together. The dyes that are used
usually respond in the visible frequency range, thus limiting the resolution of
separate points to around \SI{300}{\nano\metre}.

There are several techniques that have been developed to avoid this problem.
Two examples that are used in medical imaging are PALM\cite{owen2010palm} and
STORM\cite{rust2006sub}. These both employ special dyes that allow molecules
to be imaged at different times, a subset of all the dye molecules is imaged at
a time. Using the point spread function (PSF) of the imaged molecule, the
precise location can be estimated and combined with the locations of other
molecules from other images of the cell. This allows the diffraction limit to
be circumvented and produces a large number of points, each representing a
single molecule attached to a position on the cell.

\section{Project}
\label{sec:project}

This project aims to interpret the data that is produced from this analysis in
a more efficient way, and to provide certain quantitative statistics regarding
the data; size of structure, number of points, density, etc. Current analysis
methods have proven to be inefficient and are often not able to cope well with
noisy data. This project will investigate alternative methods for identifying
clusters in the data and ignoring erroneous points.

\subsection{Uniform Discrete Cell Method}
\label{sub:uniform_discrete_cell_method}

An initial method for identifying structure in the data points will involve
splitting the image space into discretized cells and treating each cell as a
grey scale pixel. The value that will be assigned to each pixel will depend on
the number of data points that it contains and noise will be removed by
thresholding the image to a predetermined limit value.

\subsection{Quadtree Method}
\label{sub:quadtree_method}

A second method that will be investigated will be to use a quadtree abstract
data type to hold the data points. Each datum will be placed into the quadtree
such that a maximum number of points in each node is allowed. This should have
the benefit of being much more dynamic, so respond better to the types of
structure expected.

With both of these initial approaches, a method for classifying the resultant
information will need to be produced and tested.

\section{Deliverables}
\label{sec:deliverables}

The final goal deliverable will be an easily usable and intuitive plugin for
the ImageJ program. ImageJ\cite{rasband1997imagej}, developed by the Natural
Institutes of Health, is used as the industry standard for analysis and
manipulation of biological or medical images. Since it is an public-domain
program with an open Java plugin architecture, this should be achievable.

\section{Software Development Model}
\label{sec:software_development_model}

Since there is a distinction in steps between the initial design and development
of the strategy to use to analyse the data and writing the plugin for ImageJ,
the steps involved in this project will follow an agile development model.
There will be stages of development of the algorithms and the plugin and these
will be revisited as necessary during the development.

\section{Preliminary Timescale}
\label{sec:preliminary_timescale}

\begin{center}
	\begin{tabu} to \pagewidth {c X c}
	\toprule
	Stage & Tasks & Date \\
	\midrule
	1 & Research existing methods of cluster analysis and identify short
	  comings. & 20th June 2014 \\
	1 & Build implementation of Uniform Discrete Cell method. & 24th June 2014\\
	1 & Build implementation of Quadtree method. & 30th June 2014 \\
	1 & Test and compare previous algorithms. Perform timing and resource
	  usage analysis. & 8th June 2014 \\
	1 & Using chosen method, implement cluster analysis algorithms. & \\
	1 & Write-up of background research and current implementations
	  investigation & \\
	1 & Write-up of data structure algorithms. & \\
	1 & Write-up of cluster analysis algorithms. & \\
	1 & Final write-up of processes, improvements and end results of project. &
	  \\
	\bottomrule
	\end{tabu}
\end{center}
