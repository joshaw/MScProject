% Created:  Fri 01 Aug 2014 02:02 PM
% Modified: Fri 01 Aug 2014 02:11 PM
% @author Josh Wainwright
% filename: analysis.tex

\section{Cluster Analysis}
\label{sec:cluster_analysis}

% TODO intro to cluster analysis section

\subsection{Quadtree Node Analysis}
\label{sub:quadtree_node_analysis}

% TODO explain different between node alaysis and point analysis

\subsubsection{Cluster Area}
\label{ssub:Cluster Area}

To calculate the area of the clusters, each node in each cluster is examined.
Since the size of the node must be known, it is calculated from the quadtree
code for that node. The formula is shown in Equation~\ref{eq:node-area},
\begin{align}
	a &= \frac{1}{4^{d}} \\
	a_i &= 4^{-l_i/2}, \label{eq:node-area}
\end{align}
where $a_i$ is the area of the node $i$, $d$ is the depth in the quadtree and
$l_i$ is the length of the quadtree code of that node. For every node in the
cluster, where there are $n$ nodes, this value is summed to give the total
cluster area, $A$:
\begin{align}
	A &= \sum_{i=0}^{n} a_i.
\end{align}

\subsubsection{Cluster Perimeter}
\label{ssub:Cluster Perimeter}

Similarly to the cluster area, the perimeter is given as a fractional value of
the length of one side of the whole image, as calculated for each node from the
quadtree code, as shown in Equation~\ref{eq:node-perimeter},
\begin{align}
	p &= \frac{1}{2^{d}} \\
	p_i &= 2^{-l_i/2}, \label{eq:node-perimeter}
\end{align}
where the symbols have the same meaning as above.

However, this simply gives the length of one side of the node for any node. In
order to calculate the perimeter of the cluster, it is not enough to simply
sum these values, as for the cluster area, since not all nodes contribute to
the perimeter. Instead, for each node, it must be decided whether it
contributes to the perimeter and how much (1, 2, 3 or 4 sides), and then
increase the total perimeter by this many times the length of one side. The
total perimeter, $P$, then is given by Equation~\ref{eq:total-perimeter},
\begin{align}
	P &= \sum_{i=0}^{n} s * p_i, \label{eq:total-perimeter}
\end{align}
where $s \in \{0..4\}$. This is demonstrated in
Figure~\ref{fig:perimeter-edges} where the perimeter is simple to calculate in
case a) as the nodes are all the same, but the size of each node must be taken
into account in case b).

\begin{figure}[tbhp]
	\centering
	\begin{subfigure}[c]{3.5cm}
		\includegraphics[width=\textwidth]{perimeter-edges-grid.pdf}
		\caption{}\label{fig:perimeter-edges-grid.pdf}
	\end{subfigure}%
	\quad
	\begin{subfigure}[c]{3.5cm}
		\includegraphics[width=\textwidth]{perimeter-edges-quadtree.pdf}
		\caption{}\label{fig:perimeter-edges-quadtree.pdf}
	\end{subfigure}

	\caption{When calculating the perimeter of a cluster using the nodes that
		contribute, the size of each node must be taken into account. In
		case~\subref{fig:perimeter-edges-grid.pdf}, the process is simple since
		all nodes are the same size and a fractional value of 3.5 is
		calculated. For case~\subref{fig:perimeter-edges-quadtree.pdf}, the
		steps are 25 lengths of size $\rfrac{1}{8}$ and 6 lengths of size
		$\rfrac{1}{16}$ which gives the same result, 3.5.}
	\label{fig:perimeter-edges}

\end{figure}

Within the cluster, \emph{holes} occur where a node, or number of nodes, is
surrounded on all sides by the same cluster. An example can be seen in
Figure~\ref{fig:kernel-options}. Unfortunately, these are included in the
calculation of the perimeter and so, where holes exist in a cluster, the actual
perimeter is slightly smaller than that calculated.

\subsubsection{Cluster Roundness}
\label{ssub:Cluster Roundness}

% TODO roundness writing
\begin{figure}[tbhp]
	\centering
	\begin{subfigure}[b]{4.2cm}
		\fbox{\includegraphics[width=\textwidth]{roundness-long.png}}
		\caption{}\label{fig:roundness-long.png}
	\end{subfigure}%
	\quad
	\begin{subfigure}[b]{4.2cm}
		\fbox{\includegraphics[width=\textwidth]{roundness-round.png}}
		\caption{}
	\end{subfigure}
	% TODO caption
	\caption{}\label{fig:roundness-round.png}
\end{figure}

\subsection{Point Analysis}
\label{sub:point_analysis}

\subsubsection{Cluster Area}
\label{ssub:Cluster Area}

\subsubsection{Cluster Perimeter}
\label{ssub:Cluster Perimeter}

\cite{lee2002polygonization}

\cite{estivill2000autoclust}

\cite{xia2006border}
