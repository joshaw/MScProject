\documentclass{beamer}
\usepackage[latin1]{inputenc}
\usepackage{graphicx}
% \usepackage[table]{xcolor}
\usepackage{xcolor,colortbl}

\definecolor{lyellow}{HTML}{FCE94F}
\definecolor{lorange}{HTML}{FCAF3E}
\definecolor{lbrown}{HTML}{E9B96E}
\definecolor{lgreen}{HTML}{8AE234}
\definecolor{lblue}{HTML}{729FCF}
\definecolor{lpurple}{HTML}{AD7FA8}
\definecolor{lred}{HTML}{EF2929}
\definecolor{silver}{HTML}{D3D7CF}
\definecolor{lgrey}{HTML}{888A85}

\usetheme{Pittsburgh}

\title{Cluster Analysis in Medical Imaging Data}
\author{Josh Wainwright}
\institute{Supervisor: Iain Styles}
\date{September 3rd, 2014}

\begin{document}

\graphicspath{ {images/} }

\begin{frame}
\titlepage%
\end{frame}

\begin{frame}{Introduction}
	\begin{enumerate}
		\item Background
		\item Specification
		\item Implementation---Example
	\end{enumerate}
\end{frame}

\begin{frame}{Background}
	\begin{itemize}
		\item Sub-diffraction limit imaging: PALM, STORM\@.
		\item Allows researchers to avoid diffraction limit.
		\item Abbe's criterion:
			\begin{align}
				d &= \frac{\lambda}{2n\sin\theta}
			\end{align}
		\item Objects smaller than $d$ cannot be resolved.
	\end{itemize}
\end{frame}

\begin{frame}{Background}
	\begin{enumerate}
		\item Attach markers to the object to be imaged,
			\begin{itemize}
				\item ``Photoswitchable fluorophores'', Cy5.
			\end{itemize}
		\item\label{step} Turn some markers `on', leaving others `off',
		\item Light incident on the markers is absorbed and re-emitted at a
			known frequency,
		\item Take an image of those active markers,
		\item Turn all markers `off'
		\item Goto step~\ref{step}
	\end{enumerate}
\end{frame}

\begin{frame}
	\begin{figure}
		\includegraphics[height=0.8\textheight]{storm-1.pdf}
	\end{figure}
\end{frame}

\begin{frame}
	\begin{figure}
		\includegraphics[height=0.8\textheight]{storm-2.pdf}
	\end{figure}
\end{frame}

\begin{frame}
	\begin{figure}
		\includegraphics[height=0.8\textheight]{storm-3.pdf}
	\end{figure}
\end{frame}

\begin{frame}
	\begin{figure}
		\includegraphics[height=0.8\textheight]{storm-4.pdf}
	\end{figure}
\end{frame}

\begin{frame}
	\begin{figure}
		\includegraphics[height=0.8\textheight]{storm-5.pdf}
	\end{figure}
\end{frame}

\begin{frame}
	\begin{figure}
		\includegraphics[height=0.8\textheight]{storm-6.pdf}
	\end{figure}
\end{frame}

\begin{frame}
	\begin{figure}
		\includegraphics[height=0.8\textheight]{storm-7.pdf}
	\end{figure}
\end{frame}

\begin{frame}
	\begin{figure}
		\includegraphics[height=0.8\textheight]{storm-8.pdf}
	\end{figure}
\end{frame}

\begin{frame}{Specification}
	\begin{itemize}
		\item Plugin for ImageJ
		\item Allow
			\begin{itemize}
				\item selecting file,
				\item adjust parameters,
				\item draw results to screen.
			\end{itemize}
	\end{itemize}

\end{frame}

\begin{frame}
	\begin{figure}
		\includegraphics[width=\textwidth]{complete-tree.pdf}
	\end{figure}
\end{frame}

\begin{frame}
	\begin{figure}
		\includegraphics[width=0.8\textwidth]{quadtree-tree.pdf}
	\end{figure}
\end{frame}

\begin{frame}{Example Clustering}
	\begin{table}
		\begin{tabular}{l l}
			\cellcolor{lred}   & Node included in cluster \\
			\cellcolor{lblue}  & Node ready to be checked \\
			\phantom{one}      & Node not yet examined \\
			\cellcolor{silver} & Node excluded from clusters \\
		\end{tabular}
	\end{table}
\end{frame}

\begin{frame}
	\begin{figure}
		\includegraphics[width=0.8\textwidth]{quadtree-prop1.pdf}
	\end{figure}
\end{frame}

\begin{frame}
	\begin{figure}
		\includegraphics[width=0.8\textwidth]{quadtree-prop2.pdf}
	\end{figure}
\end{frame}

\begin{frame}
	\begin{figure}
		\includegraphics[width=0.8\textwidth]{quadtree-prop3.pdf}
	\end{figure}
\end{frame}

\begin{frame}
	\begin{figure}
		\includegraphics[width=0.8\textwidth]{quadtree-prop4.pdf}
	\end{figure}
\end{frame}

\begin{frame}
	\begin{figure}
		\includegraphics[width=0.8\textwidth]{quadtree-prop5.pdf}
	\end{figure}
\end{frame}

\begin{frame}
\end{frame}

\begin{frame}{Cluster Statistics}
	\begin{figure}
		\includegraphics[width=\textwidth]{hulls.pdf}
	\end{figure}
\end{frame}

\begin{frame}
\end{frame}

\end{document}
