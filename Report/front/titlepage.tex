% Created:  Sun 15 Jun 2014 04:29 pm
% Modified: Sun 15 Jun 2014 04:29 pm

\begin{titlepage}
	\begin{center}
		\vspace*{\fill}

		\centering
		\includegraphics[scale=1.0]{Logo.pdf}
		\vfill

		\hrule
		{\LARGE\bf Medical Image Processing with Quadtrees\\[0.4cm]}
		\hrule

		\vfill
		\large
		School of Computer Science\\
		University of Birmingham

		\vfill
		Josh Wainwright
		\vfill

		\vfill
		\textit{Supervisor:} Iain Styles \\
		\vfill
		\textit{Date:} September 2014
		\vfill
		\vfill

	\end{center}
\end{titlepage}

\onecolumn

\part*{Abstract}
\label{prt:abstract}
\addcontentsline{toc}{section}{Abstract}

This study deduces that reionization began at a redshift of $z=17.82(+3.06,
-2.4)$ and ended at a redshift of $z=7\pm 1.8$. This is calculated by directly
applying the dynamics of star formation and the ionization rate of neutral
hydrogen in the Inter-galactic Medium. A photometry strategy consisting of 3
multi-band surveys is proposed in order to observe Lyman Break Galaxies across
redshifts 6--17. The surveys will locate $100.5\pm37.0$, $138.7\pm 100.6$,
$358.1\pm 158.6$ galaxies in redshift ranges 6--8.5, 8.5--10 and 10--17
respectively.  These surveys will be completed by the James Webb Space
Telescope and Euclid which are planned for launch in the coming decade. A
follow up spectroscopy survey will be used to confirm the redshift and
properties of 24, 4 and 48 galaxies in these 3 surveys respectively. The
spectroscopy will be carried out using James Webb Space Telescope and a
combination of single and multi-slit spectroscopy. It is shown that the use of
known gravitational lenses, located between redshift 0.5--0.7, is very
beneficial for discovering high redshift candidates as it can increase the
depth of surveys by up to 3 magnitudes.

\clearpage

\tableofcontents
\addcontentsline{toc}{section}{Table of Contents}

\twocolumn
