% Created:  Tue 31 Jun 2014 04:42 PM
% @author Josh Wainwright
% File name : imagej.tex

\part{ImageJ Plugin}
\label{prt:imagej_plugin}

ImageJ is a Java based image manipulation program written and maintained by
developers at the National Institute of Health which is in the public domain
and is open sourced. It is widely used in medical and biological research and
has an open API to allow extension via macros, plugins and scripts.

Since they offer significantly better integration, meaning speed and efficiency
improvements, a plugin is chosen to integrate this project into ImageJ. A
plugin was chosen over macros, since these often suffer performance loss when
more than a few steps are involved; and chosen over scripts since these do not
offer such tight integration with the rest of the program.

The plugin shall allow a user to
\begin{enumerate*}[label=\itshape\alph*\upshape)]
	\item load a data set, as gathered from STORM, PALM or similar imaging
		techniques discussed in
		Section~\ref{sec:sub_diffraction_limit_imaging};
	\item choose the format of the data (the columns that are of interest
		etc.);
	\item analyse the data for clusters; and
	\item receive detailed information regarding the clusters that were found.
\end{enumerate*}
