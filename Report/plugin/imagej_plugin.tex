% Created:  Fri 04 Jul 2014 04:47 PM
% Author:   Josh Wainwright
% Filename: imagej_plugin.tex

\section{ImageJ Plugin}
\label{sec:imagej_plugin}

The main deliverable for this project is a plugin for extracting clusters of
points from large data sets, for the image processing program ImageJ. This
plugin is written in Java and makes use of the ImageJ Java API\cite{imagejapi}.
Each of the following sections describes a part of the plugin. The general flow
of activities from opening the plugin is:

\begin{enumerate}
	\item click the \texttt{Data File} button,
	\item choose the data file to analyse,
	\item If the default separator is incorrect, input the correct one and
		click the \texttt{Read File} button to re-read with the new separator,
	\item select the values for the x- and y-columns,
	\item adjust the sliders to select the parameters that are used to build
		the quadtrees,
	\item adjust the check boxes to change the way the final image is
		displayed,
	\item click the \texttt{Quadtree} button to start the process of reading
		and analysing the selected file,
	\item use the additional information presented in the results table.
\end{enumerate}

\subsection{Column Picker}
\label{sub:column_picker}

The first step is for the user to select a data file. This is performed with
the Java built-in file choose. To make the plugin more general purpose than
being limited to only data formatted in precisely the same way as the data in
the sample files used during development, the ability to specify some
formatting features of the data file used are available. The user can specify
the delimiter used in the file (how the columns are separated; comma, space tab
etc.), and which column represents the x- and y-coordinates. These are chosen
through a separate graphical user interface (GUI).

\subsection{Option Sliders}
\label{sub:option_sliders}

The creation of the quadtrees is subject to a number of parameters that affect
the way points are added, and how it manages the propagation once all points
are added. A summary of the options available and what effect they have is
given below. Each heading represents the label given to the corresponding
slider in the GUI\@.

\begin{description}

	\item[Density] As points are added to a leaf in the quadtree, the capacity
		of the leaf is checked against this value. If the number of points in
		that leaf exceeds this value, the points are removed from the leaf,
		four new quadtrees are created, and each of the points is added in to
		the appropriate new leaf, this is known as \emph{re-leafing}.

		An extremely large value for the \textbf{density}, greater than
		$\rfrac{1}{4}$~times the number of points in the data file, would
		result in only the first level of nodes being populated (since the root
		does not hold points) and the final quadtree having only 4 leaves. A
		value of 1 would mean that there would be the same number of leaf nodes
		as points in the file. This can lead to out-of-memory errors in some
		cases where the file is large and a large value for \textbf{depth
		range} is set.

		A value of the order of 20 is recommended as this reduces the number of
		quadtrees created by 10, but still provides enough resolution for
		propagation.

		\begin{tabular}{r l}
			\texttt{default} & 20 \\
			\texttt{minimum} & 1 \\
			\texttt{maximum} & 100 \\
		\end{tabular}

	\item[Depth Range] This controls how far up the quadtree will be searched
		when looking for neighbours as described in
		Section~\ref{sub:choosing_neighbours}. All neighbours that are deeper
		in the tree than the node being propagated are included irrespective of
		this setting. A large value for the \textbf{depth range} will mean that
		the propagation will go too far up the tree and, ultimately, will reach
		the root, thus automatically including the whole tree in the located
		cluster. A value of 0 means that only nodes that are in the same level
		as the current node, or deeper, are included.

		The \textbf{depth range} should always be considerably less than the
		\textbf{max depth}.

		\begin{tabular}{r l}
			\texttt{default} & 3 \\
			\texttt{minimum} & 0 \\
			\texttt{maximum} & 25 \\
		\end{tabular}

	\item[Max Depth] As well as being controlled by the number of points in the
		leaf, adding points to the quadtree is also limited by the \textbf{max
		depth}. If the depth of a node reaches the \textbf{max depth}, then it
		is not re-leafed even if the number of points in the node exceeds the
		\textbf{density}.  This allows a high value to be set for the
		\textbf{density} so that where the points are relatively clustered, the
		tree is created deep enough to distinguish them, but to prevent the
		depth getting very high for regions of localised, abnormally high
		clustering that might skew the results.

		A small value for \textbf{max depth} means that the tree will never get
		very deep. If there is noise in the points, this can cause the depth of
		the noise and the depth of the clusters to be too similar and so mean
		clustering is not effective. Setting it to a high value can effectively
		remove the constraint, as the tree is unlikely to get that deep.

		Since the number of nodes in the quadtree increases exponentially with
		the depth, a large value for the \textbf{max depth} can lead to memory
		errors, though for typical data sets, it is unusual to exceed a depth
		of around 16. As long as this depth is localised to small areas, this
		should not be a problem. A depth of 16 provides a maximum of
		almost 5 billion leaf nodes, so even extremely large files can be
		represented to a high degree of accuracy.

		\begin{tabular}{r l}
			\texttt{default} & 10 \\
			\texttt{minimum} & 5 \\
			\texttt{maximum} & 25 \\
		\end{tabular}

\end{description}

The following are settings that affect the way the clusters are displayed once
they are found.

\begin{description}
	\item[Cluster Size]\label{it:cluster-size} Once the clustering algorithm
		has completed, this value places a limit on the minimum size of a
		cluster. Any clusters that are smaller than this will not be drawn in
		the image and will not have an entry in the results table. This can be
		useful for a noisy data set that contains many small clusters which
		make the interesting clusters less easy to see. The value is given as a
		fraction of the overall size of the image, so a value for the
		\textbf{cluster size} of 0.05 means that only clusters which are larger
		than~$\rfrac{1}{20}$ of the size of the whole image are displayed.

		A value of 0 means that no clusters are hidden.

		\begin{tabular}{r l}
			\texttt{default} & 0 \\
			\texttt{minimum} & \num{5e-4} \\
			\texttt{maximum} & \num{2e-2} \\
		\end{tabular}

	\item[Scale] \label{it:scale} Internally, the points read from the file are
		stored exactly as they are read. The pixel grid for the image is then
		generated based on these values. If the maximum x- and y-coordinate are
		both \num{10000}, with a \textbf{scale} value of 1, the image would be
		\num{10000} pixels by \num{10000} pixels. Since each pixel requires a
		number of type \emph{float} to hold the RGB colour information, this
		would be far too large to draw (giving an image with a size of around
		\SI{10}{\giga\byte}). Instead, the size of each coordinate is
		multiplied by the \textbf{scale} value to reduce the overall size of
		the image.  Because this uses integer division, there is a certain
		degree of error that is introduced by this process. This can be seen
		when viewing the quadtree structure of a deep node where the size of
		neighbouring nodes at the same level do not appear to be the same size.
		This alteration is performed only on the pixel information, the
		original data is maintained and used for all subsequent operations.

		A default value of \num{0.03} is used. This gives an uncompressed image
		size of around \SI{11}{\mega\byte} for a data set with maximum x- and
		y-coordinates of \num{50000}.

		\begin{tabular}{r l}
			\texttt{default} & \num{0.03} \\
			\texttt{minimum} & 0 \\
			\texttt{maximum} & 0.1 \\
		\end{tabular}
\end{description}

The following are purely display settings. Each is a boolean value controlled
with a checkbox component. Changing any of these settings simply requires the
data to be redrawn, no further computation is performed.

\begin{description}

	\item[Lines] The quadtree can be displayed on screen to check the settings
		that have been used, above, are most appropriate. This setting toggles
		the display of the quadtree lines so that the nodes can be seen.

		\begin{tabular}{r l}
			\texttt{default} & true \\
		\end{tabular}

	\item[Points] Similarly to \textbf{Lines}, \textbf{Points} simply controls
		whether the data points from the data file are drawn as single pixels
		or not at all.

		If this and \textbf{Lines} are both set to false, then only the clusters
		themselves, as identified by the nodes that are part of the cluster,
		are drawn.

		\begin{tabular}{r l}
			\texttt{default} & true \\
		\end{tabular}

	\item[Colourize] By default, the clusters are drawn behind the lines of the
		quadtree and behind the points from the data file. The colour of each
		cluster is different from the previous cluster. There are 20 colours
		defined, meaning that when the number of clusters exceeds this number,
		the colours will repeat. The colourisation is simply to allow the
		clusters to be visually distinguished from each other. This setting
		stops the clusters from being colourized, instead using the same grey
		for all.

		\begin{tabular}{r l}
			\texttt{default} & true \\
		\end{tabular}

\end{description}

\subsection{Displaying the QuadTree}
\label{sub:displaying_the_quadtree}

The first version of the plugin simply allowed the user to visualise the data,
once it had been processed and entered into a quadtree. Though this provided
little benefit to the researcher producing data, it can be a useful tool to get
an insight into the process that a program is using to analyse one's data so
that the results can be better interpreted. For this reason, this version
served as a foundation for later versions of the plugin so that, when loading
data, a user gets the opportunity to view the data before proceeding with the
analysis.

Again, earlier versions of the plugin displayed this data using the built-in
GUI classes in the AWT\cite{zukowski1997java} and Swing\cite{loy2002java}
libraries included in the standard Java distribution. This effectively
prohibited any further actions being performed on the image once it had been
generated since what was displayed was only modifiable by the JVM via compiled
code. It was also very memory intensive since, in many cases, many thousands of
separate objects (data points represented by zero length lines, quadtree cells
by boxes, etc.) and so was slow to draw initially and redraw with any
subsequent move or resize of the window.

The code used to generate this view of the data was modified to make use of the
image generation functions present in ImageJ. Now, instead of many different
objects being manipulated for each view of the data, a single array with a
value for each pixel is needed. For the cases where the user wishes to view the
clusters that have been found, but not have them affect the image, the image is
created with a number of different \emph{slices} in the image \emph{stack},
Figure~\ref{fig:imagej-stack}. Slices are ImageJ's representation of images
with two or more layers, or alternative views, each of which resides in a stack
of slices. Each slice in a stack must have the same dimensions.

% \begin{figure}[tbh]
% 	\centering
% 	\includegraphics[width=7cm]{imagej-stack.pdf}

% 	\caption[Layers in ImageJ image representation.]{An ``image'' in ImageJ can
% 		be composed of a number of layers, each of which is easily viewable and
% 		can be saved on its own. These layers are used to display different
% 		parts of the generated image: data points, located clusters, quadtree
% 		structure, etc.}\label{fig:imagej-stack}
% \end{figure}

\subsection{Results Table}
\label{sub:results_table}

In addition to the image of the data with the clusters, the plugin also
calculates a number of statistics for each of the clusters that are found.
These include the number of points that are included in the cluster, the area
of the cluster, and it's perimeter. The area and perimeter values are given as
a fraction where the whole image represents 1. Thus, to find the actual area
for the real life object, the area of the cluster should be multiplied by the
area of the image, and for the perimeter, the length of one side of the image
should be multiplied by the perimeter.

To calculate each of these values, each node that exists in a cluster is
examined and its contribution to the overall value added. This is discussed
further in Section~\ref{ssub:cluster_perimeter_node}.

\subsection{Macro Support}
\label{sub:macro_support}

As mentioned on page~\pageref{part:imagej_plugin}, ImageJ allows extension via
macros, as well as plugins and scripts. The ImageJ documentation describes
macros as follows:

\begin{quote}
	A macro is a simple program that automates a series of ImageJ
	commands.

	[\,\ldots]

	Use the macro language's built-in \texttt{run(command,
	options)} function to run any of the 400+ commands in the ImageJ menu
	bar.  The first argument (\texttt{command}) is any ImageJ menu command
	(e.g., ``Measure'' and ``Close All'') and the second argument
	(\texttt{options}), which is optional, is a string containing parameter
	values.\\
	\hspace*{\fill}\cite{imagejapi}
\end{quote}

This means that the process of image processing can be simplified by writing or
recording macros which perform repetitive tasks automatically.

In order to make use of these features of ImageJ, support for macros must be
written into plugins. The plugin written for this project includes basic
support for macros through the \texttt{run(command, options)} command described
above. The command is ``Cluster Analysis'' and the supported options are
described below.

\tabulinesep=1.2mm
\begin{tabu}{c c X}
	\toprule
	Option & Value & Description \\
	\midrule
	\texttt{sep} & \textbf{String} & Specifies the separator to use when
		reading the data file. If this option is not given, the separator
		defaults to a single tab character. \\
	\texttt{filepath} & \textbf{String} & The data file to read from. \\
	\texttt{quadtree} & \textbf{0 or 1} & If 1, the quadtree generated from
		the given file is drawn immediately. If 0, or if this option is not
		given, the main dialogue window is opened. If 0, or not given, but a
		filepath is given, the file is read and initial analysis performed once
		the window is open, but the image is not displayed. \\
	\texttt{grid} & \textbf{0 or 1} & Same as for \texttt{quadtree} above
		but for the simple grid method. \\
	\bottomrule
\end{tabu}

Each of these options must be given in the form \texttt{option=value} where
\texttt{value} has the type or limitations described above and each
option/value pair must be separated with a comma and an optional space. All
options except \texttt{filepath} are case-insensitive.
