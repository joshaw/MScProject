% Created:  Fri 04 Jul 2014 04:47 PM
% Modified: Fri 11 Jul 2014 11:15 AM
% @author Josh Wainwright
% File name : imagej_plugin.tex

\section{ImageJ Plugin}
\label{sec:imagej_plugin}

The first version of the plugin simply allowed the user to visualise the data,
once it had been processed and entered into a quadtree. Though this provided
little benefit to the researcher producing data, it can be useful to simply get
an insight into the process that a program is using to analyse one's data so
that the results can be better interpretted. For this reason, this version
served as a foundation for later versions of the plugin so that, when loading
data, a user gets the opportunity to view the data before proceding with the
analysis.

The analytical steps that are undertaken are descibed in more detail in later
sections.

\begin{figure*}[tbhp]
	\centering
	\includegraphics[angle=90,height=\textheight]{plugin_class_diagram.pdf}
	\caption{Name}
	\label{fig:name}
\end{figure*}

\begin{figure}[tbhp]
    \centering
    \begin{subfigure}[c]{0.48\linewidth}
        \includegraphics[width=\textwidth]{single-cluster.png}
        \caption{}\label{fig:single-cluster-points}
    \end{subfigure}%
    \quad
    \begin{subfigure}[c]{0.48\linewidth}
        \includegraphics[width=\textwidth]{single-cluster-lines.png}
        \caption{}\label{fig:single-cluster-lines}
    \end{subfigure}
	% TODO caption
    \caption{} \label{fig:single-cluster}
\end{figure}
