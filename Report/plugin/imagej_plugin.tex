% Created:  Fri 04 Jul 2014 04:47 PM
% Modified: Wed 06 Aug 2014 12:24 pm
% @author Josh Wainwright
% File name : imagej_plugin.tex

\section{ImageJ Plugin}
\label{sec:imagej_plugin}

The main deliverable for this project is a plugin for extracting clusters of
points from large data sets, for the image processing program ImageJ. This
plugin is written in Java and makes use of the ImageJ Java API\cite{imagejapi}.
\subsection{Column Picker}
\label{sub:column_picker}

To make the plugin more general purpose than being limited to only data
formatted in precisely the same way as the data in the sample files used during
development, the ability to specify some formatting features of the data file
used are available. The user can specify the delimiter used in the file (how
the columns are separated; comma, space tab etc.), and which column represents
the x- and y-coordinates. These are chosen through a separate GUI.


\subsection{Displaying the QuadTree}
\label{sub:displaying_the_quadtree}

The first version of the plugin simply allowed the user to visualise the data,
once it had been processed and entered into a quadtree. Though this provided
little benefit to the researcher producing data, it can be a useful tool to get
an insight into the process that a program is using to analyse one's data so
that the results can be better interpreted. For this reason, this version
served as a foundation for later versions of the plugin so that, when loading
data, a user gets the opportunity to view the data before proceeding with the
analysis.

Again, earlier versions of the program displayed this data using the built-in
GUI classes in the AWT\cite{zukowski1997java} and Swing\cite{loy2002java}
libraries included in the standard Java distribution. This effectively
prohibited any further actions being performed on the image once it had been
generated since what was displayed was only modifiable by the JVM via compiled
code. It was also very memory intensive since, in many cases, many thousands of
separate objects (data points represented by zero length lines, quadtree cells
by boxes, etc.) and so was slow to draw initially and redraw with any
subsequent move or resize of the window.

The code used to generate this view of the data was modified to make use of the
easy image generation functions present in ImageJ. Now, instead of many
different objects being manipulated for each view of the data, a single array
with a value for each pixel is needed. For the cases where the user wishes to
view the clusters that have been found, but not have them affect the image, the
image is created with a number of different \emph{slices} in the image
\emph{stack}, Figure~\ref{fig:imagej-stack}. Slices are ImageJ's representation
of images with two or more layers, or alternative views, each of which resides
in a stack of slices. Each slice in a stack must have the same dimensions.

\begin{figure}[tbhp]
	\centering
	\includegraphics[width=7cm]{imagej-stack.pdf}
	\caption{An ``image'' in ImageJ can be composed of a number of layers, each
		of which is easily viewable and can be saved on its own. These layers
		are used to display different parts of the generated image: data
		points, located clusters, quadtree structure, etc.}
	\label{fig:imagej-stack}
\end{figure}

\subsection{Results Table}
\label{sub:results_table}

In addition to the image of the data with the clusters, the plugin also
calculates a number of statistics for each of the clusters that are found.
These include the number of points that are included in the cluster, the area
of the cluster, and it's perimeter. The area and perimeter values are given as
a fraction where the whole image represents 1. Thus, to find the actual area
for the real life object, the area of the cluster should be multiplied by the
area of the image, and for the perimeter, the length of one side of the image
should be multiplied by the perimeter.

To calculate each of these values, each node that exists in a cluster is
examined and its contribution to the overall value added.
