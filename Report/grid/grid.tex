% Created:  Mon 23 Jun 2014 13:15 PM
% @author Josh Wainwright
% File name : grid.tex

\section{Simple Grid Method}
\label{sec:simple_grid_method}

The simplest method for analysing the distribution of points is to use a
regular grid of cells and place the points into the cells one at a time. Once
all points have been added, the number of points per cell can be treated as a
grey scale brightness value. This gives a simple pixel image, with brightness
as a function of density of the points, in the \texttt{pnm} image format. A
thresholding filter can then be applied to remove the points that are isolated
and leave the denser areas corresponding to clusters.

Though the resolution of this method can be easily changed by altering the size
of the cells and the grid, it performs badly when presented with data that is
even slightly noisy. If the clusters themselves have a density that is not
significantly above the background noise level, the thresholding step is prone
to either exclude much of the real data, or to increase the size of the
clusters by including too much noise. These two effects can be seen clearly in
Figure~\ref{fig:grid-noise}, where \texttt{palm-1.txt} is used with a cell
size of 200. The range of the data is from 0 to 41000 for both the $x$ and the
$y$ axes, thus the images are 205 by 205 pixels. This data took
\SI{495}{\milli\second} to generate.

\begin{figure}[tbhp]
	\centering
	\begin{subfigure}[b]{4.2cm}
		\frame{\includegraphics[width=\textwidth]{grid-noise-low.png}}
		\caption{}\label{fig:grid-noise-low.png}
	\end{subfigure}%
	\quad
	\begin{subfigure}[b]{4.2cm}
		\frame{\includegraphics[width=\textwidth]{grid-noise-high.png}}
		\caption{}\label{fig:grid-noise-high.png}
	\end{subfigure}

	\caption[Effect of threshold value on clusters identified.]{Setting a low
		threshold, \subref{fig:grid-noise-low.png}, means that many of the
		points in the clusters are lost. Setting it higher,
		\subref{fig:grid-noise-high.png}, includes too many of the points
		deemed to be noise. Pixels are cells that ended with points, red are
		points that would be kept by the threshold process, black would be
		removed.}\label{fig:grid-noise}
\end{figure}

There are steps that can be taken to improve the approach of this simple grid
when handling outlying points caused by noise.
\begin{enumerate}

	\item First the algorithm is modified to include a thresholding step before
		writing the data to a file. This means that the pixels can be adjusted
		with greater accuracy and any arbitrary level can be chosen to
		threshold at.

	\item Next, once an image has been generated, the number of points that
		contributed to each pixel is no longer of interest and so the image can
		be converted to a binary image. This is an image with just two possible
		values, the first represents white space, where there are no points,
		the seconds is black where there were points and so is of interest.

	\item Once a binary image has been generated, erode and dilate filters can
		be applied to remove remaining outliers and to try to close the gaps in
		the structures that have been identified so that they are more solid.

\end{enumerate}

\begin{figure}[tbhp]
	\centering
	\frame{\includegraphics[width=7cm]{grid-threshold-close.png}}

	\caption[Closing algorithm to identify clusters.]{Using a
		\texttt{close}-ing algorithm can help to emphasise the structure in the
		data and, at the same time, remove the isolated points representing
		noise.}\label{fig:grid-threshold-close}
\end{figure}

These steps lead to significantly better isolation of the interesting parts of
the image, as can be seen in Figure~\ref{fig:grid-threshold-close}, however,
much of the detail of the structure is lost in this process.
