% Created:  Mon 23 Jun 2014 04:06 PM
% Modified: Sun 10 Aug 2014 11:12 am
% @author Josh Wainwright
% File name : file-columns.tex

\section{Data File Structure}
\label{app:data_file_structure}

The data files that are produced from the initial analysis of the images have a
standard format.

\begin{enumerate}
	\item Tab separated fields.
	\item Single header line with names of fields.
	\item One or more item of data, separated by newlines.
\end{enumerate}

The columns that represent fields in the file are as follows.

\begin{center}
	\begin{tabu}{p{1.1cm} X c}
		\toprule
		Header & Meaning & Used? \\
		\midrule
		\texttt{Channel Name} & Wavelength channel that was used to capture data.
			First value, $I$, is the incident wavelength of the light used to
			excite the dye and the second, $E$, is the wavelength emitted that
			was imaged. & no \\
		\texttt{X} & x-coordinate of the point & no \\
		\texttt{Y} & y-coordinate of the point & no \\
		\texttt{Xc} & centered, normalised x-coordinate of point & yes \\
		\texttt{Yc} & centered, normalised y-coordinate of point & yes \\
		\texttt{Height} & the height of the fitted gaussian peak used to
			extract the point from the original image & not yet \\
		\texttt{Area} & area of the point & not yet \\
		\texttt{Width} & full width half maximum of the point & not yet \\
		\texttt{Phi}          &  & no \\
		\texttt{Ax}           &  & no \\
		\texttt{BG}           &  & no \\
		\texttt{I}            &  & no \\
		\texttt{Frame}        &  & no \\
		\texttt{Length}       &  & no \\
		\texttt{Valid}        &  & no \\
		\texttt{Z}            &  & no \\
		\texttt{Zc}           &  & no \\
		\texttt{Photons}      &  & no \\
		\texttt{Lateral Localisation Accuracy}     &  & no \\
		\texttt{Xw}           &  & no \\
		\texttt{Yw}           &  & no \\
		\texttt{Xwc}          &  & no \\
		\texttt{Ywc}          &  & no \\
		\bottomrule
	\end{tabu}
\end{center}
