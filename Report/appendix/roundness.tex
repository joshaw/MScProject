% Created:  Sat 02 Aug 2014 06:48 pm
% @author Josh Wainwright
% File name : roundness.tex

\chapter{Roundness Derivation}
\label{app:roundness_derivation}

Derivation of the equation for roundness, $R$, of a cluster given the area,
$A$, and perimeter, $p$, of the cluster.

A circle is defined to have a roundness of 1.

\begin{align}
	R_\textrm{circle} = 1 \\
\intertext{The value must be unit-less, so divide area by perimeter squared,}
	A_{\textrm{circle}} &= \pi r^2 \\
	p_{\textrm{circle}} &= 2\pi r \\
	\frac{A}{p^2} &= \frac{\pi r^2}{4\pi^2 r^2} \\
\intertext{Remove constants to give formula,}
	R &= 4\pi \times \frac{\pi r^2}{4\pi^2 r^2} \\
		&= \frac{4\pi A}{p^2}
\end{align}

Test for regular shapes.

\paragraph{Circle:}
\label{par:circle}

\begin{align}
	A &= \pi r^2 \\
	p &= 2\pi r \\
	R &= 4\pi \times \frac{\pi r^2}{{(2\pi r)}^2} = 1
\end{align}

\paragraph{Square:}
\label{par:square}

\begin{align}
	A &= h^2 \\
	p &= 4h \\
	R &= 4\pi \times \frac{h^2}{{(4h)}^2} = \frac{\pi}{4} \approx 0.785
\end{align}

\paragraph{Equilateral Triangle:}
\label{par:equalateral_triangle}

\begin{align}
	A &= \frac{\sqrt{3}}{4}a^2 \\
	p &= 3h \\
	R &= 4\pi \times \frac{\frac{\sqrt{3}}{4}a^2}{{(3a)}^2} = \frac{\sqrt{3}\pi}{9} \approx 0.605
\end{align}

\paragraph{Ellipse ($\textrm{Eccentricity} = 0.5$):}
\label{par:elipse}

For ellipse, $\epsilon = 0.5$, let minor axis, $a = 1$ and major axis, $b = 2$.
\begin{align}
	A &= 6.28 \\
	p &= 9.69 \\
	R &= \frac{6.28}{9.69} \approx 0.648
\end{align}
