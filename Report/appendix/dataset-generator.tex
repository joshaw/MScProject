% Created:  Thu 04 Sep 2014 05:57 PM
% Author:   Josh Wainwright
% Filename: dataset-generator.tex

\newgeometry{voffset=-1cm, textheight=27cm}
\chapter{Data Set Generator}
\label{sec:dataset_generator}

A small C++ utility was written to aid the generation of data sets for testing.
This program takes an image file and produces data based on the pixels in the
image. The input must have the same format as when saving from ImageJ in the
\texttt{text image} format. The file will then be formed of a decimal value,
from 0 to 255, for each pixel in the image, separated by tab characters.

The utility code is shown in Listing~\ref{lst:dataset-gen}. It is used by
calling \texttt{gen-coords FILENAME} at the command line and outputs the
coordinates to standard output.

\begin{center}
\begin{minipage}{\textwidth}
	\begin{lstlisting}[caption={Utility to gen data sets from images.},
	label=lst:dataset-gen, language=c++, linewidth=12cm, firstnumber=1] public
	#include <iostream>
	#include <fstream>
	#include <string>
	#include <sstream>
	#include <stdlib.h>

	using namespace std;

	int main(int argc, char* argv[]) {

		if (argc != 2) {
			cout << "Usage: " << argv[0] << " FILENAME" << endl;
			exit (EXIT_FAILURE);
		}

		string filename = argv[1];
		ifstream datafile(filename.c_str());
		string line = "";

		// For each line, for each tab separated field, output coord if 0.
		int y = 0;
		while(getline(datafile, line)) {

			string delimiter = "\t";
			size_t pos = 0;
			string token;
			int x = 0;
			while ((pos = line.find(delimiter)) != string::npos) {

				token = line.substr(0, pos);
				if (token == "0") {
					cout << x << "\t" << y << endl;
				}
				x++;
				line.erase(0, pos + delimiter.length());
			}

			y++;
		}
	}
\end{lstlisting}
\end{minipage}
\end{center}
