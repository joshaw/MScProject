% Created:  Mon 25 Aug 2014 02:12 pm
% Author:   Josh Wainwright
% Filename: running-plugin.tex

\chapter{Running the Plugin}
\label{cha:running_the_plugin}

\section{Contents of CD}
\label{sec:contents_of_cd}

The files contained on the CD accompanying this report are as follows
(\folder~represents a directory and \file~represents a file):
\begin{itemize}
	\item[\file] \texttt{report.pdf}: PDF version of this report,
	\item[\folder] \texttt{plugin}: Directory containing nessessary file to
		compile and run the plugin.
		\begin{itemize}
			\item[\folder] \texttt{src}: Directory containing the Java source
				files for the ImageJ plugin.
			\item[\file] \texttt{build.xml} Apache Ant build file.
		\end{itemize}
	\item[\folder] \texttt{sampledata}: Directory containing a number of sample
		data files for testing the plugin.
\end{itemize}

\section{Compiling and Running the Plugin}
\label{sec:running_the_plugin}

The steps required to compile the plugin are given below. These steps require
Apache Ant to be installed with a relatively recent version of the JavaVM\@.
\begin{enumerate}
	\item Edit the \texttt{build.xml} file to change the third line with the
		\texttt{imagej-plugin-location} variable to point to the ``jars''
		folder within the plugin section of the ImageJ executable folder.
	\item Run the command \texttt{ant compile} from the same folder as the
		\texttt{build.xml} file to compile the Java source and generate the
		Java jar file ready for installation.
	\item If there were no errors, run the command \texttt{ant imagej} to copy
		the generated jar file to the relevant folder to be run by ImageJ.
	\item The command \texttt{ant imagej} can be run to both compile and copy
		the file.
\end{enumerate}

To run the plugin from ImageJ, follow the steps below. Note that a minimum
ImageJ version of \texttt{v1.48} is required to run the plugin. ImageJ can be
updated by clicking on \mbox{\texttt{Help >> Update ImageJ}}. The most recent
version (as of 5th Sept.\ 2014) is \texttt{v1.49g}.
\begin{enumerate}
	\item If ImageJ is already running, refresh the menus so that the latest
		version of the plugin by clicking \texttt{Help >> Refresh Menus}.
	\item Run the plugin by clicking \texttt{Plugins >> jars >> Cluster
		Analysis}.
\end{enumerate}
